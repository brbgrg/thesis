Topological measures that significantly change with age across the lifespan

lifespan trajectory of the topological organization of human whole-brain functional networks

characterize topological age-related effects in the global and regional organization of the human brain
functional connectome across the lifespan

\subsection{functional connectivity topological measures}

We employed R-fMRI data and graph-theorymethods to
systematically characterize topological age-related effects
in the global and regional organization of the human brain
functional connectome across the lifespan (7–85 years).

We systematically analyzed the global, regional and
connectional properties (for illustrations, see Fig. 1) of the
resultant brain functional networks as follows.

(i) To examine the global network properties, we first
explored the topological efficiency (local and global), modularity and 
vulnerability/robustness of the brain networks, which are
described briefly below.

high local efficiency,which indicates highly clustered connections between 
topologically
nearby neighbors, has been suggested to be correlated with
efficient information processing among functional specialized regions as 
well as high error tolerance (Bullmore and
Sporns, 2009; He and Evans, 2010; Sporns et al., 2005).

Besides examining the absolute network properties based on the 
correlation thresholding method, we
also examined the relative network properties based on
density thresholding networks considering that different
numbers of edges in networks could interfere with detecting 
age-related differences in topological organization (He
et al., 2009; Rubinov and Sporns, 2010)

To characterize the age effects on the global
network topological properties, three key graphic 
metrics were employed, network efficiency (global efficiency
and local efficiency) and modularity, which were all 
calculated based on both ‘absolute’ and ‘relative’ networks. We
found significant age-related differences in network local
efficiency and module structure. The absolute global 
efficiency of brain functional networks showed no significant
relationship with age, while the local efficiency showed
an inverted U-shaped trajectory (absolute global 
efficiency: p = 0.13, r2 = 0.12; local efficiency: p = 0.03, r2 = 0.21;
Fig. 2A). Although all individual brain networks 
exhibited significantly modular structures across the lifespan
(all Z-scores > 113.3, p < 0.0001), the absolute modularity
decreased linearly with age (p < 0.001, r2 = 0.15; Fig. 2A).
Notably, similar age-related differences were observed for
both relative global and local efficiency (relative global 
efficiency: p = 0.13, r2 = 0.12; local efficiency: p = 0.09, r2 = 0.14;
Fig. 1B). The relative modularity also decreased linearly
with age (p < 0.001, r2 = 0.22; Fig. 2B). The modularity number 
did not differ significantly with age for either the
absolute or the relative functional networks

Significant sex differences were observed for
many global properties including global efficiency
(male > female, absolute: p = 0.001, relative: p < 0.001),
local efficiency (male > female, absolute: p = 0.01, relative:
p = 0.005; Figure S5)


(ii) To explore nodal properties, we considered the rFCS
(i.e., the weighted degree centrality)

Regions with higher (>1 SD beyond the mean) rFCSs,
referred to as hubs, are thought to play important roles in
the communication of information in brain connectomics

In this study, we further explored the connectivity patterns
among these functional hubs. We considered the hubs and
their connections as a sub-network and calculated the correlation 
density using the weighted rich club coefficient
measure.  

A significantly greater rich club coefficient in a brain network
than expected by chance (i.e., the results of random networks) indicates 
the existence of a rich club organization.

In other words, the hub nodes are more densely connected
among themselves than non-hub nodes and thus form a
highly interconnected club.

normalized rich club coefficient

To evaluate the reproducibility of rich club analysis, we
also used four other thresholds to define the hub regions
(rFCS > 0.5, 0.75, 1.25, 1.5 SD beyond the mean)

Notably, both the mean rFCS map across all subjects and the fitted rFCS maps of different 
age populations were highly similar to the hub probability map derived
from all subjects.

Further analyses revealed that the brain hubs were densely
connected: the weighted rich club coefficients, ˚, of the
sub-network composed of brain hubs were significantly
larger than those of matched random networks, ˚random
(all Z-scores > 20.51 ps < 0.001). In addition, an increasing
mean ˚norm over a range of hub thresholds was observed
(Fig. 4A). This result suggests that the functional networks
of the human brain contained a rich club architecture in
which the highly connected regions were more densely
linked among themselves than the weakly connected
regions. Fig. 3C shows the rich club structure of the group
mean brain network. Notably, the rich club phenomenon
has been recently demonstrated in the human brain structural 
connectome (Collin et al., 2013; van den Heuvel et al.,
2012; van den Heuvel and Sporns, 2011). Intriguingly, the
normalized rich club coefficient ˚norm showed inverted
U shaped lifespan trajectories (p = 0.01, r2 = 0.10; Fig. 4B),
indicating that the brain’s functional rich club architecture increased 
until approximately 40 years of age and
decreased at older ages. These findings persisted over a
range of hub thresholds (Figs. 4A and S3). The 3D surface
visualizations of the results were implemented using the
Brain Net Viewer (www.nitrc.org/projects/bnv) (Xia et al.,
2013).
We also identified the brain regions showing significant
age-related changes in the rFCS across the lifespan (Fig. 5,
p < 0.05, FDR-corrected). Linear age-related decreases in
rFCS were predominantly located in several default-mode
regions (bilateral medial prefrontal cortex), attention
regions (bilateral insula), visual cortex (bilateral middle
occipital gyrus and right calcarine) and subcortical regions
(bilateral putamen and left caudate; Fig. 5A). Part of the
left precuneus showed a linear decreasing trend with age
(p = 0.01, uncorrected; Figure S4 A). The rFCS of the left supplementary 
motor area, the right inferior temporal gyrus
and the left temporal pole increased significantly with age
(Fig. 5A). The positive quadratic (U-shaped) trajectories of
rFCS with age were mainly located in the parahippocampus and 
thalamus (Fig. 5B). Negative quadratic age effects
(inverted-U shaped) were found in the lateral frontal, parietal 
and temporal regions (inferior frontal gyrus, precentral
gyrus, postcentral gyrus, inferior parietal gyrus, rolandic
operculum, middle temporal gyrus and inferior temporal
gyrus), and cuneus (Fig. 5B). Bilateral middle frontal gyrus,
medial superior frontal gyrus, and left intraparietal sulcus all 
showed a trend of negative quadratic age effects
(p < 0.01, uncorrected; Figure S4 B). Most of these age-related 
regions were identified as hub regions. The rFCS
maps for every decade are shown in Fig. 3C.

For nodal properties, the
males showed higher connectivity strength in the left
supplementary motor area, insula and bilateral putamen 
(p < 0.05, FDR corrected). Only the cerebellum crus1
showed a significant sex and age interaction effect (p < 0.05,
FDR corrected)


(iii) To explore the properties of functional network
connectivity, we employed three measures: network connection density; 
network mean connectivity strength; and network mean anatomical distance.

the network mean connectivity strength
(p = 0.006, r2 = 0.28) and the network mean anatomical
distance (p < 0.001, r2 = 0.11) followed negative quadratic
trajectories over the age range.

The distance associated connection analyses revealed that the proportions 
of short-distance connections showed U-shaped
trajectories with age with peak ages at approximately 40
years (10–60 mm, ps < 0.02, r2 = 0.13 ± 0.03), while the proportions 
of long-distance connections showed inverted
U-shaped trajectories with peaks around 45 years of age
(70–140 mm, p < 0.01, r2 = 0.11± 0.038; Fig. 7A). It should
be noted that for both short- and long-distance connections, when the 
connections were longer, the peak
ages were older.

The correlation strengths of both short- and long-distance connections 
exhibited negative quadratic age-related changes
(20–160 mm, p < 0.02, r2 = 0.13 ± 0.03); however, the connections shorter 
than 20 mm decreased linearly with age
(p < 0.01, r2 > 0.17; Fig. 7B)

The ‘absolute’ thresholding approach (Hayasaka and
Laurienti, 2010; van den Heuvel et al., 2008, 2009),
which only preserves the significantly existing correlations, 
produced individual brain networks containing
different edge numbers, resulting in a connection density
range of 5–20%
Of the 126 networks, 83 were fully connected; 
the remaining had at least 99.0% 
of their nodes
fully connected. Although age and density were unrelated
(r = −0.08, p = 0.35), we employed a second ‘relative’ 
threshold method in which networks were constructed with
the same density threshold value to have the same edge
numbers

Comparable results were obtained for other two low resolution parcellation 
schemes. No significant age-related
differences in network density were detected for either
template (L-Dos: p = 0.18; L-Yeo: p = 0.52). Furthermore,
all main topological findings obtained using the high
resolution template were reproduced under at least
one low-resolution functional template (Fig. 8). Some
parcellation-based differences emerged. Specifically, we
observed linearly decreasing global and local efficiency
under the L-Dos template and preserved modularity and
connectivity strength under the L-Yeo template (Fig. 8)


\cite{Cao2014}


\subsection{limitations}
Several studies have demonstrated that the quantification of topological 
organization of brain networks is
parcellation-dependent (de Reus and van den Heuvel,
2013; Wang et al., 2009). 



\cite{Cao2014}


\subsection{discussion}

In this view, the inverted-U shaped change trajectory
of local efficiency supports the notion that functionally related 
regions or segregated functional processing
systems emerge during development (Fair et al., 2009),
optimize during adulthood and deteriorate with aging
(Meunier et al., 2009a)
\cite{Cao2014}


Significantly denser connections between hub regions compared
with non-hub regions form the rich club organization,
which is high-cost but provides significant functional benefits by 
enhancing not only global information flow but
also the resilience of the network to hub attacks. In this
study, we observed the existence of rich club organization
in the functional network over the lifespan. Interestingly,
the rich club structure demonstrated significantly negative
quadratic age effects. As the rich club organization makes
important contributions to interregional information traffic 
and cognitive values in healthy populations (Bullmore
and Sporns, 2012; van den Heuvel et al., 2012), changes
with inverted U-shapes of this core architecture may correlate 
with high cognitive function changes over the lifespan
Similar lifespan trajectories of the
rich club organization and the strength and proportion of
long-distance connections may indicate that communication within these 
rich club regions plays a central role
in long-distance brain communication and in optimizing
global brain communication efficiency for healthy cognitive brain 
functioning
\cite{Cao2014}



