\subsection{Temporal and Spatial Resolution Limitations}
The proposed method could take advantage of the high spatial 
resolution of fMRI and high temporal resolution of
MEG to achieve the highest sex classifcation performance of 85.2%.
\cite{Zhao2022}

Diffusion magnetic resonance imaging (dMRI) and functional MRI reveals the large
scale SC across different brain regions. Electrophysiological methods (i.e.
MEG/EEG) provide direct measures of neural activity and exhibits complex 
neurobiological temporal dynamics which could not be solved by
fMRI. However, most of existing multimodal analytical methods collapse the brain 
measurements either in space or time domain and fail
to capture the spatio-temporal circuit dynamics. In this paper, we propose a novel 
spatio-temporal graph Transformer model to integrate the
structural and functional connectivity in both spatial and temporal domain.

The proposed method is evaluated with the meta-analysis to explore the
behavioral relevance of different brain regions and characterize the brain
dynamical organization into low level functions region (i.e. sensory) and the
complex function regions (i.e. memory).
\cite{Zhao2022a}

\subsection{Pipeline limitations}
While effective, this model is not trained in an end-to-end
fashion since the clustering, cluster-specific CBT generation, 
and the population CBT estimation blocks are learned
disjointly. Therefore, cumulated errors across these blocks
might produce a less centered brain template
\cite{Bessadok2022}

\subsection{Interpretability Limitations}
To highlight the importance of the brain ROIs, we introduce the explainable
causal representation to encourage the reasonable node selection process. We
train an explanation model to explain the multmodal graph representation approach based on granger causality.
\cite{Zhao2022}