
\section{Dataset Preprocessing}
Functional networks are rendered as temporal correlation matrices, and anatomical networks were converted into structural 
correlation matrices.
The transformation applied to SC networks is fundamental for this analysis and in general for those analysis aimed at 
comparing SC and FC topological organization. 
If we had preserved SC networks in the form of a sparse positive matrix with
weights possibly much higher than 1 (i.e., maximum weight reachable in FC networks), the community detection process would  
have been biased: depending it on the weights of the matrices, it would have found modules reflecting almost exclusively 
the anatomical modular organization.
Moreover, this transformation allowed us to use the same mathematical instruments (i.e., the same spatial null model in 
the optimization) on both matrices.
\cite{Puxeddu2022}.

\section{Structural and Functional Connectivity Matrices}
Structural connectivity (SC) matrices represent the anatomical connections between different brain regions.
Functional connectivity (FC) matrices represent the temporal correlations between brain regions during resting state.

Most existing studies focus more on structural brain networks 
and functional brain networks, where both brain networks are 
undirected attributed graphs with undirected edge eij between 
vi and vj (i.e., eij ¼ eji). The difference between structural
networks and functional networks is that the e 2 E in structural 
networks are positive values, while they can be negative in 
functional networks.
Since the functional networks are constructed based on the BOLD 
signal correlation among different brain nodes, the positive and 
negative edges represent synchronous activation and asynchronous 
activation among brain regions, respectively. \cite{Tang2023}.

The blood oxygenated level-dependent (BOLD) signal of functional MRI
(fMRI) reveals the spatial and temporal brain activity across diferent
brain regions \cite{Zhao2022}. 

\section{Dataset Description}
The dataset used in this study comprises structural connectivity (SC) and functional connectivity (FC) matrices 
obtained from MRI and fMRI scans, respectively. The data is categorized into three age groups: 
young (10 years old), adult (30 years old), and old (70 years old). Each age group contains data from 5 different subjects.
Both SC and FC matrices have dimensions of 200 $\times$ 200 $\times$ 5 (nodes $\times$ nodes $\times$ subjects).

Nodes are already aligned across subjects and modalities based on a 
common parcellation scheme, so no graph matching is needed? \cite{Puxeddu2022}
