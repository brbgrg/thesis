
\subsection{Motivation}
The human cerebral cortex is organized into functionally segregated 
neuronal populations connected by the anatomical pathways. 
White matter fiber tracts form a connectome of structural 
connectivity at the macroscale. This structural connectome exhibits 
a complex network topology characterized by non-random properties, 
including small-world architecture2, segregated 
communities3, and a core of densely inter-connected hubs4. 
These topological patterns support the communication dynamics on 
structural networks and coordinate the temporal synchronization of 
neural activity—termed functional connectivity—between cortical regions5-8.

Understanding how structural connectivity shapes functional connectivity 
patterns is central to neuroscience. 

higher structure-function coupling has been related to better performance 
in executive function10,22, and abnormal patterns of the
coupling are associated with a wide range of psychiatric and 
neurological disorders, such as major depressive disorder26, 
bipolar disorder27, attention deficit hyperactivity disorder28, 
and Parkinson’s disease29.

\cite{Chen2024}


\subsection{ Dataset Challenges}
Another challenge of the current brain network
dataset is data insufficiency, which will further 
limit the progress of big data mining on brain network studies. 
For example, the current brain network datasets may not be easy 
to utilize for the group difference studies based on the deep 
learning model since the number of networks in a few subgroups 
may not be enough to train the neural networks.
Instead of enlarging the current dataset, technical methods 
in addressing data quantity issues should also be strictly 
considered. These methods include but are not limited to 
data augmentation techniques, fast algorithms for brain 
network constructions from neuroimaging data, multisite 
learning for dataset combinations, and pre-trained model
development. \cite{Tang2023}