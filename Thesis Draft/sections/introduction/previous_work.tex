\subsection{Techniques used}



\subsection{Expected Findings}

Expected age-dependent modular reorganization based on
the findings of \cite{Puxeddu2022} are weaker SC-FC coupling, 
greater reconfiguration of FC modules, 
and increased inter-subject variability in older adults 
compared to younger individuals.

1. Age-Related Changes in Structure-Function Coupling:
Reduced Coupling with Aging: Older subjects may exhibit a weaker coupling between 
SC and FC networks compared to younger subjects. This is consistent with findings that 
brain networks become less synchronized as individuals age. In particular, structural 
connections may degrade over time, while functional connections maintain higher variability, 
leading to a decoupling of SC and FC modules.
Young and Adult Groups: In contrast, young and adult subjects might show stronger SC-FC coupling, 
where structural modules are more predictive of functional modules.
2. Modular Reconfiguration with Aging:
Functional Networks: The functional networks of older individuals are likely to show greater 
reconfiguration over time. Age-related changes in cognitive function often manifest in more 
distributed and variable modular organization, especially in functional networks. You might 
observe that the same brain regions participate in different modules across age groups, 
especially in older adults, where functional modules are less stable compared to younger adults.
Structural Networks: Structural modularity tends to remain more consistent, even in aging 
populations, but age-related atrophy can lead to disruptions in connectivity for older adults. 
This means that modular consistency in SC networks might decline with age.
3. Variability Across Subjects in Older Adults:
The inter-subject variability in both SC and FC networks is expected to increase with age. 
Older individuals may show more subject-specific modular patterns, as both SC and FC networks 
are likely to deviate more from the typical patterns seen in younger or middle-aged adults.
4. Spatial Specificity of Changes:
Certain brain systems, such as the default mode network (DMN), dorsal attention network (DAN), 
and somatomotor areas, may undergo age-related changes in coupling between SC and FC. 
These areas are known to show functional reconfiguration with aging, while structural connections 
might become more localized or weakened.
5. Temporal Dynamics:
In older adults, temporal fluctuations in FC are expected to become more prominent, especially 
during resting state, due to the reduced synchrony between SC and FC. You may observe temporal 
dependencies where older adults' FC networks fluctuate more frequently than younger individuals', 
reflecting compensatory mechanisms or diminished structural support for stable functional 
connectivity.

