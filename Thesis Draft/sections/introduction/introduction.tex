
\section{Introduction}
It is widely accepted that the human brain has one of the
most complex networks known to humanity and acts as the
biological hardware to control our cognition and behaviors.
Recently, advanced noninvasive neuroimaging techniques,
e.g., magnetic resonance imaging (MRI), have revealed
brain functional activities and anatomical structures in
vivo. One of the modern approaches in neuroscience is
to consider brain region interactions as a graph network,
referred to as brain connectome or brain network (Kong and
Yu 2014; Chung 2019) 

Brain network analysis has been
demonstrated to be effective for the diagnosis and prognosis
of neurological disorders (Cao et al. 2015; Tadayonnejad
and Ajilore 2014; Wang et al. 2017).



\cite{Zhang2022}.


data: functional and structural connectivity matrices from fMRI and MRI. 

task: 
- cross-modality and cross-subject modular comparison 
to detect age-related modular differences, highlighting shared 
and unique brain network components across groups. 
-  understanding the relationship between brain structure and function 
across different age groups

focus: 
- age-related modular reorganization: changes in how SC and FC networks are partitioned 
    into modules as people age
- age-related differences in SC-FC coupling: how the relationship between structural and 
    functional modules differs between younger and older individuals
- inter-subject variability in modular structure: subject-specific modular organization 
    and how it changes with age
\cite{Puxeddu2022}

research objective: 
dynamic brain network learning? \cite{Tang2023}, 
clustering
state-of-the-art deep-learning methods for brain network mining
framework using Graph Neural Networks (GNN) for structure-function coupling analysis

GNN variant: variant with maximum expressing power that performs
well on the task (state-of-the-art baseline with greater test-set
accuracy)

DL Model Selection: Graph Neural Network (GNN) model designed for 
community detection and modular comparison

Model Architecture (layers, activation functions, etc.):



